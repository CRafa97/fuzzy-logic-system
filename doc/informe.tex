%===================================================================================
% PREÁMBULO
%-----------------------------------------------------------------------------------
\documentclass[a4paper,10pt,twocolumn]{article}

%===================================================================================
% Paquetes
%-----------------------------------------------------------------------------------
\usepackage{amsmath}
\usepackage{amsfonts}
\usepackage{amssymb}
\usepackage{informe}
\usepackage{lipsum}
\usepackage[utf8]{inputenc}
\usepackage{listings}
\usepackage{algorithmic}
\usepackage[pdftex]{hyperref}
%-----------------------------------------------------------------------------------
% Configuración
%-----------------------------------------------------------------------------------
\hypersetup{colorlinks,%
	    citecolor=black,%
	    filecolor=black,%
	    linkcolor=black,%
	    urlcolor=blue}

%===================================================================================



%===================================================================================
% Presentacion
%-----------------------------------------------------------------------------------
% Título
%-----------------------------------------------------------------------------------
\title{Sistema de Inferencia Difusa}

%-----------------------------------------------------------------------------------
% Autores
%-----------------------------------------------------------------------------------
\author{\\
	\name Carlos Rafael Ortega Lezcano \\ \addr Grupo C411 }


%-----------------------------------------------------------------------------------
% Tutores
%-----------------------------------------------------------------------------------
%\tutors{\\}

%-----------------------------------------------------------------------------------
% Headings
%-----------------------------------------------------------------------------------
%\jcematcomheading{\the\year}{1-\pageref{end}}{Carlos Rafael}

%-----------------------------------------------------------------------------------
%\ShortHeadings{Simulacio\'n basada en Eventos Discretos}{Carlos Rafael}
%===================================================================================



%===================================================================================
% DOCUMENTO
%-----------------------------------------------------------------------------------
\begin{document}

%-----------------------------------------------------------------------------------
% NO BORRAR ESTA LINEA!
%-----------------------------------------------------------------------------------
\twocolumn[
%-----------------------------------------------------------------------------------

\maketitle

%===================================================================================
% Resumen y Abstract
%-----------------------------------------------------------------------------------
\selectlanguage{spanish} % Para producir el documento en Español

%-----------------------------------------------------------------------------------
% Palabras clave
%-----------------------------------------------------------------------------------
%\begin{keywords}
%	Separadas,
%	Por,
%	Comas.
%\end{keywords}

%-----------------------------------------------------------------------------------
% Temas
%-----------------------------------------------------------------------------------
%\begin{topics}
%	Tema, Subtema.
%\end{topics}


%-----------------------------------------------------------------------------------
% NO BORRAR ESTAS LINEAS!
%-----------------------------------------------------------------------------------
\vspace{0.8cm}
]
%-----------------------------------------------------------------------------------


%===================================================================================

%===================================================================================
% Introducción
%-----------------------------------------------------------------------------------
\section*{Introducci\'on}\label{sec:intro}
%-----------------------------------------------------------------------------------

El sistema de inferencia difusa implementado presentar\'a funciones de pertenencia triangulares, trapezoidales, gaussianas y sigmoidales ademas cuenta con los metodos de agregacion de Mamdani y Larsen, con entrada de valores \textit{singleton} o sea valores precisos o como entrada conjuntos difusos, los metodos de desdifusificaci'on empleados son el Centroide (COA), Bisecci\'on (BOA) y en el caso de los M\'aximos, el MOM o Media de los M\'aximos.

Para validar dicho sistema se resolver\'a un problema cuya soluci\'on necesita de un Sistema de Inferencia y se analizaran los resultados obtenidos. 

\section*{Caracter\'isticas del Sistema de Inferencia }

El sistema implementado se compone primeramente por las reglas \textit{if-then} que definen el problema a resolver, los m\'etodos de agregaci\'on de Mamdani y Larsen y los m\'etodos de desdifusicificaci\'on.

\subsection*{Funciones de Pertenencia}

Las funciones de pertenencia m\'as conocidas las cuales fueron implementadas para comodidad a la hora de emplear el sistema fueron:

\begin{enumerate}
	\item \textbf{Triangulares}: Es un conjunto difuso representado por 3 puntos $A = (a_1, a_2, a_3)$ y su funci\'on de pretenencia es:
		\begin{align*}
			\mu_A (x) = \begin{cases}
				0, & x < a_1 \\\\
				\dfrac{x - a_1}{a_2 - a_1}, & a_1 \leq x \leq a_2 \\\\
				\dfrac{a_3 - x}{a_3 - a_2}, & a_2 \leq x \leq a_3 \\\\
				0, & x > a_3
			\end{cases}
		\end{align*}
		
	\item \textbf{Trapezoidales}: Es un conjunto difuso representado por 4 puntos, $A = (a_1, a_2, a_3, a_4) $ y su funci\'on de pretenencia es:
		\begin{align*}
		\mu_A (x) = \begin{cases}
			0, & x < a_1 \\\\
			\dfrac{x - a_1}{a_2 - a_1}, & a_1 \leq x \leq a_2 \\\\
			1, & a_2 \leq x \leq a_3 \\\\
			\dfrac{a_4 - x}{a_4 - a_3}, & a_3 \leq x \leq a_4 \\\\
			0, & x > a_4
		\end{cases}
		\end{align*}
	
	\item \textbf{Gaussiana}: Es un conjunto difuso cuya funci\'on de pertenecia es una funci\'on exponencial definida por dos valores $k$ y $m$, su representacion corresponde con la campana de Gauss:
		\begin{align*}
			\mu_A (x) = e^{-k(x-m)^{2}}
		\end{align*}
	
	\item \textbf{Sigmoidal}: Es un conjuto difuso cuya funci\'on de pertenencia es presenta un crecimiento m\'as lento q una parte de una funci\'on triangular o trapezoidal, esta definida por dos valores $a$ y $b$ y el valor $m$, el cual usualmente es $m = \dfrac{a + b}{2}$, su funci\'on de petenencia es:
		\begin{align*}
			\mu_A (x) = \begin{cases}
				0, & x \leq a \\\\
				2\left[ \dfrac{x - a}{b - a}
				\right]^{2}, & a < x \leq m	 \\\\
				1 - 2 \left[
				\dfrac{x - b}{b - a}
				\right]^{2}, & m < x < b \\\\
				1, & x \geq b
			\end{cases}
		\end{align*}
\end{enumerate}

\subsection*{M\'etodos de Agregaci\'on y Desdifusificaci\'on}

El sistema emplea los m\'etodos de Mamdani y Larsen para determinar una agregaci\'on, se determinan los valores de los $\alpha_i$ dependiendo del tipo de entrada y luego se determina la funci\'on de pertenencia para la agregaci\'on $C'$:
	\begin{enumerate}
		\item[] \textbf{Mamdani}: $\mu_{C'} (z) = \bigvee_{i=1}^{n} \left[ \alpha_i  \wedge \mu_{C_i} (z) \right]$ 
		
		\item[] \textbf{Larsen}: $\mu_{C'} (z) = \bigvee_{i=1}^{n} \left[ \alpha_i \cdot \mu_{C_i} (z) \right]$
	\end{enumerate}

Para la desdifusificaci\'on del conjunto resultante se implementar\'on 3 variantes:

	\begin{enumerate}
		\item[] \textbf{Media de los M\'aximos}: Representa el promedio de aquellos valores de control $z_j$ donde se alcanza el m\'aximo:
			\begin{align}
				z_0 = \sum_{j=1}^{k} \dfrac{z_j}{k}
			\end{align}
		
		\item[] \textbf{Centroide}: Esta estrategia genera el centro de gravedad de los conjuntos que conforman la agregaci\'on y se define como:
			\begin{align}
				z_0 = \dfrac{\sum_{j=1}^{n} \mu_{C} (z_j) \cdot z_j}{\sum_{j=1}^{n} \mu_{C} (z_j)}
			\end{align}
			 
	\end{enumerate}

\section*{Development}\label{sec:dev}
  
\lipsum[6-8]

\section*{Conclussion}\label{sec:con}

\lipsum[9-11]

\begin{thebibliography}{9}
	
\end{thebibliography}

\label{end}

\end{document}

%===================================================================================
